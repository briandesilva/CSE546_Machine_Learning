\documentclass{article}

\usepackage{fancyhdr}
\usepackage{extramarks}
\usepackage{amsmath}
\usepackage{amsthm}
\usepackage{amsfonts}
\usepackage{amssymb}
\usepackage{graphicx}
\usepackage{caption,subcaption}
\usepackage{subfig}
\usepackage{enumerate}          % For enumerates indexed by letters
\usepackage{bm}                 % For bold letters
\usepackage{algorithm2e}        % For pseudocode
% \usepackage{hyperref}                % For urls
% \usepackage{bbold}              % For indicator function


%
% Basic Document Settings
%

\topmargin=-0.45in
\evensidemargin=0in
\oddsidemargin=0in
\textwidth=6.5in
\textheight=9.0in
\headsep=0.25in

\linespread{1.1}

\pagestyle{fancy}
\lhead{\hmwkAuthorName}
\chead{\hmwkClass:\ \hmwkTitle}
\rhead{\firstxmark}
\lfoot{\lastxmark}
\cfoot{\thepage}

\renewcommand\headrulewidth{0.4pt}
\renewcommand\footrulewidth{0.4pt}

\setlength\parindent{0pt}

\setcounter{section}{-1}




%
% Homework Details
%   - Title
%   - Due date
%   - Class
%   - Section/Time
%   - Instructor
%   - Author
%

\newcommand{\hmwkTitle}{Homework 4}
\newcommand{\hmwkDueDate}{December 12, 2016}
\newcommand{\hmwkClass}{CSE 546}
\newcommand{\hmwkAuthorName}{Brian de Silva}

%
% Title Page
%

\title{
    \vspace{2in}
    \textmd{\textbf{\hmwkClass:\ \hmwkTitle}}\\
    \normalsize\vspace{0.1in}\small{Due\ on\ \hmwkDueDate\ }\\
    \vspace{3in}
}

\author{\textbf{\hmwkAuthorName}}
\date{}


% Useful commands
\newcommand{\E}{\mathbb{E}}
\newcommand{\Var}{\mathrm{Var}}
\newcommand{\Cov}{\mathrm{Cov}}
\newcommand{\Bias}{\mathrm{Bias}}
\newcommand{\bbm}{\begin{bmatrix}}
\newcommand{\ebm}{\end{bmatrix}}
\newcommand{\R}{\mathbb{R}}
\newcommand{\X}{\mathbb{X}}
\newcommand{\Z}{\mathbb{Z}}


\begin{document}

\maketitle

\pagebreak

% Problem 0
\section{Collaborators and Acknowledgements}
% I read the post at \url{http://ufldl.stanford.edu/tutorial/supervised/MultiLayerNeuralNetworks/} when working problem two.

I collaborated with the following people
\begin{itemize}
    \item Kathleen Champion: problem three
\end{itemize}

% Problem 1
\section{Manual Calculation of one round of EM for a GMM}
Let $x=\bbm 1&10&20\ebm$ be the set of 1-D data points we are given.

\subsection{M step}
Suppose the output from the E step is the following matrix
\[
    R = \bbm 1 & 0 \\ 0.4 &0.6 \\ 0 & 1 \ebm
\]
where entry $R_{i,c}$ is the probability of observation $x_i$ belonging to cluster $c$.
\begin{enumerate}
    \item We are trying to maximize
    \begin{align*}
        \mathbb{E}[\log Pr(\mathbb{X},\Z|\Theta)] &=\mathbb{E}_{\Z\sim Pr(\Z|\X,\Theta^{(t-1)})}\left[\log Pr(\X,\Z|\Theta) \right]\\
        &=\mathbb{E}_{\Z\sim Pr(\Z|\X,\Theta^{(t-1)})}\left[\log \sum_i\left(Pr(x_i,z_i|\Theta) \right) \right]\\
        &=\sum_i\mathbb{E}_{\Z\sim Pr(\Z|\X,\Theta^{(t-1)})}\left[ \log Pr(x_i,z_i|\Theta)\right]\\
        &=\sum_i\mathbb{E}_{\Z\sim Pr(\Z|\X,\Theta^{(t-1)})}\left[\log\left[\prod_{c} \pi_cPr(x_i|\Theta_c))^{\mathbb{I}(z_i=c)}\right]\right] \\ 
        &= \sum_i\sum_{c}\mathbb{E}_{\Z\sim Pr(\Z|\X,\Theta^{(t-1)})}[\mathbb{I}(z_i=c)]\log(\pi_c Pr(x_i,\Theta_c))\\
        &=\sum_i\sum_cPr(z_i=c|x_i,\Theta^{(t-1)})\log(\pi_cPr(x_i|\Theta_c))\\
        &= \sum_i\sum_cR_{ic}(\log\pi_c + \log Pr(x_i|\Theta_c))
    \end{align*}
    over the parameters associated with each class, collected in $\Theta_c$. 
    \item Let $R_k = \sum_i R_{i,k}$. Then for $k=1,2$
    \[
        \pi_k = \frac13\sum^3_{i=1}R_{i,k} = R_k / 3.
    \]
    Hence
    \begin{align*}
        \pi_1 &= \frac13(1+0.4)=\frac{7}{15},\\
        \pi_2 &= \frac13(0+1) = \frac{8}{15}.
    \end{align*}
    \item Recall that for $k=1,2$ the means are given by
    \[
        \mu_k = \frac{\sum_{i=1}^3R_{i,k}x_i}{R_k}
    \]
    with $R_1=\tfrac{7}{5}$ and $R_2=\tfrac{8}{5}$. Hence
    \begin{align*}
        \mu_1 &= \frac{1\cdot 1+0.4\cdot10}{7/5} = \frac{25}{7},\\
        \mu_2 &= \frac{0.6\cdot10+1\cdot20}{8/5} = \frac{65}{4}.
    \end{align*}
    \item Recall that for $k=1,2$, the standard deviations $\sigma_k$ are
    \[
        \sigma_k=\frac{\sum_{i=1}^3R_{i,k}x_i^2}{R_k} - \mu_k^2.
    \]
    Therefore
    \begin{align*}
        \sigma_1 &= \sqrt{\frac{1\cdot 1 + 0.4\cdot100}{7/5} -\left(\frac{25}{7}\right)^2} = \sqrt{\frac{810}{49}},\\
        \sigma_2 &= \sqrt{\frac{0.6\cdot 100+1\cdot400}{8/5}-\left(\frac{65}{4}\right)^2} = \sqrt{\frac{375}{16}}.
    \end{align*}
\end{enumerate}

\subsection{E step}
\begin{enumerate}
    \item 
    \[
        Pr(z_i=c)=R_{i,c} = \frac{\pi^{(t-1)}_cPr\left(x_i\left|\Theta^{(t-1)}_c\right.\right)}{\sum^2_{c'=1}\pi_{c'}Pr\left(x_i\left|\Theta_{c'}^{(t-1)}\right.\right)},
    \]
    where
    \[
        Pr\left(x_i\left|\Theta^{(t-1)}_c\right.\right) = \frac{\pi_c^{(t-1)}}{\sqrt{2\pi}\sigma_c^{(t-1)}}exp\left(-\left(x_i-\mu_c^{(t-1)}\right)^2\left/2\left(\sigma^{(t-1)}_c\right)^2\right.\right).
    \]
    \item Substituting in the values $\Theta^{(t-1)}$ we obtained in the M step into the above formula, we obtain
    \begin{align*}
        Pr\left(1\left|\Theta^{(t-1)}_1\right.\right)&\approx 0.0374898 \\
        Pr\left(1\left|\Theta^{(t-1)}_2\right.\right)&\approx 0.000307803\\
        Pr\left(1\left|\Theta^{(t-1)}_1\right.\right)+Pr\left(1\left|\Theta^{(t-1)}_2\right.\right)&\approx  0.0377976\\
        &\implies R_{1,1}\approx 0.991857,~~R_{1,2}\approx 0.00814346.
    \end{align*}
    \begin{align*}
        Pr\left(10\left|\Theta^{(t-1)}_1\right.\right)&\approx 0.0131191 \\
        Pr\left(10\left|\Theta^{(t-1)}_2\right.\right)&\approx  0.0191003\\
        Pr\left(10\left|\Theta^{(t-1)}_1\right.\right)+Pr\left(10\left|\Theta^{(t-1)}_2\right.\right)&\approx  0.0322194\\
        &\implies R_{2,1}\approx 0.40718,~~R_{2,2}\approx 0.59282.
    \end{align*}
    \begin{align*}
        Pr\left(20\left|\Theta^{(t-1)}_1\right.\right)&\approx  0.0000130429\\
        Pr\left(20\left|\Theta^{(t-1)}_2\right.\right)&\approx 0.0325585 \\
        Pr\left(20\left|\Theta^{(t-1)}_1\right.\right)+Pr\left(20\left|\Theta^{(t-1)}_2\right.\right)&\approx 0.0325716\\
        &\implies R_{3,1}\approx 0.000400438 ,~~R_{3,2}\approx 0.9996.
    \end{align*}
    Putting this all together, the new value of $R$ is 
    \[
        R=\bbm 0.991857 & 0.00814346 \\
        0.40718 & 0.59282 \\
        0.000400438 & 0.9996
        \ebm.
    \]
\end{enumerate}

% Problem 2
\section{Neural Nets and Backprop}
\subsection{With tanh hidden units}
\begin{enumerate}
    \item We used the following parameter choices:
    \begin{itemize}
        \item Learning rate: I started with an initial learning rate of $10^{-3}$ which I reduced by a factor of $4$ every 10 epochs.
        \item Mini-batch size: 10.
        \item Weight initalization: I initialized the weights for the input nodes by drawing each entry from a normal distribution with mean 0 and variance $1/E[\|X\|^2]$, where $X$ is the training data matrix. The weights for the hidden nodes were drawn from a normal distribution with mean 0 and variance $1/\sqrt{50}$ (where 50 is the number of input nodes). I then obtained the model's predictions $\hat Y$ and decreased the weights by a factor of $100E[\hat Y]$ to attempt to ensure that $E[\hat Y]\leq 0.1E[Y]$ for the next round of approximations generated by the method, $\hat Y$. Note that $E[Y]=0.1$ since the original labels are converted to one-hot vectors of length 10. The method was run for 30 epochs.
        \item Other: No regularizations or offsets were used.
        
    \end{itemize}
    \item
    \begin{figure}
        \centering
        \includegraphics[width=.85\textwidth]{sqLoss_2-1-30Ep}
        \caption{The square loss on the training and test sets for a 2-layer neural network with $\tanh$ activation functions} 
        \label{fig:sqLoss_2-1}
    \end{figure}
    Figure \ref{fig:sqLoss_2-1} shows the decay of the mean square error as a function of epochs. The error is plotted every half-epoch.
    \item
    \begin{figure}
        \centering
        \includegraphics[width=.85\textwidth]{z1Loss_2-1-30Ep}
        \caption{The 0/1 loss on the training and test sets for a 2-layer neural network with $\tanh$ activation functions} 
        \label{fig:z1Loss_2-1}
    \end{figure}
    Figure \ref{fig:z1Loss_2-1} shows the decay of the 0/1 loss as a function of epochs. The error is plotted every half-epoch.
    \item My final losses are summarized in the following table

    \begin{tabular}{l|ll}
    & Square loss & 0/1 loss \\
    \hline
    Training &  0.05566 & 0.0142 \\
    Testing  &  0.074909 & 0.0249
    \end{tabular}
    \item 
    \begin{figure}
        \centering
        \includegraphics[width=.85\textwidth]{weightVisualization_2-1_30Ep}
        \caption{Visualizations of learned weights for 10 random nodes with $\tanh$ activation functions} 
        \label{fig:weights2-1}
    \end{figure}
    Figure \ref{fig:weights2-1} provides grayscale visualizations of the weights of 10 random nodes from the hidden layer of our network.
\end{enumerate}

\subsection{With ReLu hidden units}
\begin{enumerate}
    \item We used the exact same parameter configuration and initialization procedure for this problem as in the previous one.
    \item
    \begin{figure}
        \centering
        \includegraphics[width=.85\textwidth]{sqLoss_2-2-30Ep}
        \caption{The square loss on the training and test sets for a 2-layer neural network with $\tanh$ activation functions} 
        \label{fig:sqLoss_2-2}
    \end{figure}
    Figure \ref{fig:sqLoss_2-2} shows the decay of the mean square error as a function of epochs. The error is plotted every half-epoch.
    \item
    \begin{figure}
        \centering
        \includegraphics[width=.85\textwidth]{z1Loss_2-2-30Ep}
        \caption{The 0/1 loss on the training and test sets for a 2-layer neural network with $\tanh$ activation functions in the hidden layer} 
        \label{fig:z1Loss_2-2}
    \end{figure}
    Figure \ref{fig:z1Loss_2-2} shows the decay of the 0/1 loss as a function of epochs. The error is plotted every half-epoch.
    \item My final losses are summarized in the following table

    \begin{tabular}{l|ll}
    & Square loss & 0/1 loss \\
    \hline
    Training & 0.055168  & 0.009900 \\
    Testing  & 0.071822  & 0.017300
    \end{tabular}

    \item 
    \begin{figure}
        \centering
        \includegraphics[width=.85\textwidth]{weightVisualization_2-2_30Ep}
        \caption{Visualizations of learned weights for 10 random nodes with ReLu activation functions in the hidden layer} 
        \label{fig:weights2-2}
    \end{figure}
    Figure \ref{fig:weights2-2} gives a grayscale visualizations of the weights of 10 random nodes from the hidden layer of our network. These weights were taken from the same network as the previous images in this section.
\end{enumerate}

% Train Square loss after 30 epochs:   0.055168
% Train 0/1 loss after 30 epochs:      0.009900
% Test Square loss after 30 epochs:    0.071822
% Test 0/1 loss after 30 epochs:       0.017300

\subsection{With ReLu hidden units and ReLu output units}

    \begin{enumerate}
    \item We used the exact same parameter configuration and initialization procedure for this problem as in the previous one.
    \item
    \begin{figure}
        \centering
        \includegraphics[width=.85\textwidth]{sqLoss_2-3-30Ep}
        \caption{The square loss on the training and test sets for a 2-layer neural network with ReLu activation functions in the hidden and output layers} 
        \label{fig:sqLoss_2-2}
    \end{figure}
    Figure \ref{fig:sqLoss_2-2} shows the decay of the mean square error as a function of epochs. The error is plotted every half-epoch.
    \item
    \begin{figure}
        \centering
        \includegraphics[width=.85\textwidth]{z1Loss_2-3-30Ep}
        \caption{The 0/1 loss on the training and test sets for a 2-layer neural network with ReLu activation functions in the hidden and output layers} 
        \label{fig:z1Loss_2-3}
    \end{figure}
    Figure \ref{fig:z1Loss_2-3} shows the decay of the 0/1 loss as a function of epochs. The error is plotted every half-epoch.
    \item My final losses are summarized in the following table

    \begin{tabular}{l|ll}
    & Square loss & 0/1 loss \\
    \hline
    Training & 0.018621  & 0.003817 \\
    Testing  & 0.041433  & 0.014300
    \end{tabular}

    \item 
    \begin{figure}
        \centering
        \includegraphics[width=.85\textwidth]{weightVisualization_2-3_30Ep}
        \caption{Visualizations of learned weights for 10 random nodes with ReLu activation functions in the latter two layers} 
        \label{fig:weights2-3}
    \end{figure}
    Figure \ref{fig:weights2-3} gives a grayscale visualizations of the weights of 10 random nodes from the hidden layer of our network. These weights were taken from the same network as the previous images in this section.
\end{enumerate}

    % \item Params: 30 epochs, batch size 10, step size 1.e-3, no reg or offset, cut down weights by a factor of 4 every 10 epochs, same weight normalization in all cases except no renormalization of input weights
    % \item sq train:    0.018621 
    % \item 0/1 train:   0.003817
    % \item sq test:     0.041433 
    % \item 0/1 test:    0.014300

% Problem 3
\section{EM vs. Gradient Descent}


% Problem 4
\section{Markov Decision Processes and Dynamic Programming}
\begin{enumerate}
    \item Note that $\sum_{x'}Pr(x'|x,a) = 1$.
    For any two value functions $V_1,~V_2$, we have
    \begin{align*}
        \|Bell(V_1)-Bell(V_2)\|_{\infty} &= \max_s\left|\tilde V_1(s)-\tilde V_2(s)\right|\\
        &= \max_s\left|\max_a\left(R(s,a) +\gamma\sum_{x'}Pr(x'|s,a)V_1(x')\right)\right.\\&\quad\left.-\max_b\left(R(s,b) +\gamma\sum_{x'}Pr(x'|s,b)V_2(x')\right) \right| \\
        &\leq \max_s\left| \max_a\left(\gamma\sum_{x'}Pr(x'|s,a)V_1(x')-\gamma\sum_{x'}Pr(x'|s,b)V_2(x')\right)\right|\\
        &= \max_s\left|\gamma\max_a\left(\sum_{x'}Pr(x'|s,a)\left(V_1(x')-V_2(x')\right)\right)\right|\\
        &\leq \max_s\left|\gamma\max_a\left( \sum_{x'}Pr(x'|s,a)\max_{x}(V_1(x)-V_2(x))\right) \right|\\
        &= \gamma \max_s\left|\max_x(V_1(x)-V_2(x)) \max_a\left(\sum_{x'}Pr(x'|s,a)\right) \right|\\
        &= \gamma \left|\max_x(V_1(x)-V_2(x)) \right| \\
        &=\gamma \max_x\left|V_1(x)-V_2(x)\right|\\
        &= \gamma \|V_1-V_2\|_{\infty}.
    \end{align*}
    Thus the Bellman operator is a contraction mapping.
    \item Let $V$ be a fixed point of the Bellman operator, so that $Bell(V)=V$. Suppose $U$ is also a fixed point of the Bellman operator. Then by the above
    \[
        \|U-V\|_{\infty} = \|Bell(U)-Bell(V)\|_{\infty}\leq \gamma \|U-V\|_{\infty}.
    \]
    This implies that $(1-\gamma)\|U-V\|_{\infty}\leq0$. But $1-\gamma>0$ by assumption, so the only way that this can be true is if $\|U-V\|_{\infty}=0$. So we must have
    \[
        \|U-V\|_{\infty}=\max_s\|U(s)-V(s)\|=0,
    \]
    from which it follows that $U=V$. Hence $V$ must be unique.
\end{enumerate}

\end{document}

